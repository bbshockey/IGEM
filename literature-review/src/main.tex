% Preamble
\documentclass[journal = bichaw, biochem=true]{achemso}

% Packages
\usepackage{amsmath}
\usepackage{titling}
\usepackage{authblk}

\title{Literature on Modeling of Bioengineered Bacteria in Soil}

\author[1]{Bjorn Shockey}
\author[1]{Walker Knapp}
\author[1]{Zhe Liu}
\affil[1]{College of William and Mary}

% Document
\begin{document}

    % Introduction

    Comprehensive literature has been published on the distribution of bacteria in soil~\cite{TimsinaRameshChandra2021AMMf}.

    % **** Very **** Tenative Sections, these can change


    \section{Modeling Bacterial Growth in Soil}\label{sec:modeling-bacterial-growth-in-soil}
    So much of genetic engineering and synthetic biology is conducted in labs where conditions are easily controlled and measured. The
    temperature, humidity, PH levels, density of medium, nutrients, and many more variables influencing growth are all controlled as the 
    researcher deems appropriate. Additionally, the use of antibiotics ensures plasmid survival allowing scientists to more easily 
    observe their desired bacterial growth. However, in soil gathered to represent a realistic biological enviornment these conditions 
    are not held constant and miryad problems arise in modeling and measuring bacterial growth. 
    Due to the novelty of introducting engineered bacteria into natural soil there are very few papers published on modeling bacterial 
    growth. However, due to health concerns there have been studies done that measure the growth and presence of Escherichia Coli in 
    manure. In one study, researchers looked at all of the factors influincing E. Coli growth in manure and determined "Spatiotemporal 
    factors influence survival durations of E. Coli more than amendment type, total amount of E. Coli present, organic or conventional 
    soil management, and depth of manure application"~\cite{SharmaManan2019SoEc}. Understanding how different growth variables effect our 
    chassis before our manipulation is imperative to understanding growth of it post engineering. Their growth model accounted for $98$ 
    percent of variation in E. Coli growth. These researchers derived their predictive model from a machine learning algorrithm called 
    the random forrest model. In a similar paper researchers attempted to measure the accuracy of the random forrest model(RMF). They 
    observed over $300$ plots of manure and rigirously measured the growth of E. Coli over four years. They used the acceptable 
    prediction zone approach as outlined by ~\cite{OscarThomasE2005VoLT} to confrim that their RMF met their accuracy standards, which it 
    did~\cite{PangHao2020APMf}. These papers prove that by using the RMF model researchers can accurately predict the growth of E. Coli 
    in soil. However, no research has been found indicating that this appraoch has been used to model other chassis in soil. As the first 
    step to modeling our engineered bacteria we should model each chassis we use and create RMF models predicting its growth prior to 
    engineering. Then the same should be repeated with the engineering bacteria. The change in the model should indicate the effect of 
    our engineering and the difference in deathrates can be directly calculated to the success of the bacteria to consume plastics in 
    soil. 
    
    



    \section{Modeling Microbe Movement in Soil}\label{sec:modeling-microbe-movement-in-soil}

    \bibliography{main}

\end{document}
